% This file is public domain

\documentclass[wcp]{jmlr}

\usepackage{lipsum}% Provides dummy text for this example

\newcommand{\samplecommand}{\textbf{A sample command}}

\jmlrvolume{42}
\jmlryear{2010}
\jmlrworkshop{Workshop on Causality}
\jmlrpages{1-14}

\title[Causal Inference and Uplift Modelling]{Causal Inference and Uplift Modelling: a Review of the Literature}

\author{\Name{Pierre Gutierrez}\Email{bdw@sample.com}\\
\Name{Jean-Yves G{\'e}rardy}\Email{emf@sample.com}}

\begin{document}
\maketitle

\begin{abstract}
\lipsum[1]
\end{abstract}
\begin{keywords}
Sample
\end{keywords}

\section{Introduction}

This is a sample article. \sectionref{sec:method} discusses
the method used. \equationref{eq:emc2} is an interesting 
equation. The results are discussed in \sectionref{sec:results}
and some other stuff is in \appendixref{apd:first}.\footnote{Here's
a footnote.}
\samplecommand.

\lipsum

\section{Method}\label{sec:method}

\lipsum

\begin{equation}\label{eq:emc2}
E = mc^2
\end{equation}

\section{Results}\label{sec:results}

\lipsum[1-55]

\appendix
\section{First Appendix}\label{apd:first}

\lipsum
\end{document}
